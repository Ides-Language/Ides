%template for producing IEEE-format articles using LaTeX.
%written by Matthew Ward, CS Department, Worcester Polytechnic Institute.
%use at your own risk.  Complaints to /dev/null.
%make two column with no page numbering, default is 10 point
\documentstyle[twocolumn]{article}
\pagestyle{empty}

%set dimensions of columns, gap between columns, and space between paragraphs
\setlength{\textheight}{8.75in}
\setlength{\columnsep}{2.0pc}
\setlength{\textwidth}{6.8in}
\setlength{\footheight}{0.0in}
\setlength{\topmargin}{0.25in}
\setlength{\headheight}{0.0in}
\setlength{\headsep}{0.0in}
\setlength{\oddsidemargin}{-.19in}
\setlength{\parindent}{1pc}

%I copied stuff out of art10.sty and modified them to conform to IEEE format

\makeatletter
%as Latex considers descenders in its calculation of interline spacing,
%to get 12 point spacing for normalsize text, must set it to 10 points
\def\@normalsize{\@setsize\normalsize{12pt}\xpt\@xpt
\abovedisplayskip 10pt plus2pt minus5pt\belowdisplayskip \abovedisplayskip
\abovedisplayshortskip \z@ plus3pt\belowdisplayshortskip 6pt plus3pt
minus3pt\let\@listi\@listI} 

\def\code{\texttt}

%need an 11 pt font size for subsection and abstract headings
\def\subsize{\@setsize\subsize{12pt}\xipt\@xipt}

%make section titles bold and 12 point, 2 blank lines before, 1 after
\def\section{\@startsection {section}{1}{\z@}{24pt plus 2pt minus 2pt}
{12pt plus 2pt minus 2pt}{\large\bf}}

%make subsection titles bold and 11 point, 1 blank line before, 1 after
\def\subsection{\@startsection {subsection}{2}{\z@}{12pt plus 2pt minus 2pt}
{12pt plus 2pt minus 2pt}{\subsize\bf}}
\makeatother

\begin{document}

%don't want date printed
\date{}

%make title bold and 14 pt font (Latex default is non-bold, 16 pt)
\title{\Large\bf The Ides Programming Language}

%for single author (just remove % characters)
%\author{I. M. Author \\
%  My Department \\
%  My Institute \\
%  My City, ST, zip}
 
%for two authors (this is what is printed)
\author{\begin{tabular}[t]{c@{\extracolsep{8em}}c}
  Sean T. Edwards  & Advisor: Larry Latour \\
 \\
 \multicolumn{2}{c}{Department of Computer Science} \\
 \multicolumn{2}{c}{University of Maine School of Computing} \\
 \multicolumn{2}{c}{Orono, ME 04469} 
\end{tabular}}

\maketitle

%I don't know why I have to reset thispagesyle, but otherwise get page numbers
\thispagestyle{empty}

\subsection*{\centering Abstract}
%IEEE allows italicized abstract
{\em
Abstract stuff
%end italics mode
}

\section{Introduction}

Intro stuff

P.S. look, I actually used \LaTeX

\section{Rationale}
Because languages are neat.

\section{Design}
Stuff I thought about.
\subsection{Type System}

\subsection{Expression Evaluation}

\subsection{The Standard Library}

\section{\code{idesc}: The Ides Compiler}
\subsection{The Lexer}


\subsection{The Parser}


\subsection{Semantic Analysis}


\subsection{\code{Std.Lang}: Compiler Support}


\subsection{Code Generation}
LLVM makes the electronics do switchy stuff.

\section{Summary and Conclusions}
Ides is neat.

\section{Future Work}
Do the actual dang report, dangit.

\section{Related Work}
\subsection{Scala}
Scala\cite{key:scala} is a hybrid object-oriented and functional language for the Java Virtual Machine, and has informed numerous design decisions for Ides.

\subsection{Rust}
Rust is a type-safe, low-level systems programming language developed by Mozilla. Rust informs many design decisions surrounding memory management and data ownership rules. 

\subsection{F\# ("F Sharp")}

\subsection{LLVM}
LLVM is an open source compiler infrastructure. It is the underlying technology for several compilers, most notably Clang, a compiler for the C family of languages.

\subsection{Flex}
Flex is a tool for building lexical analyzers, based on the AT\&T software lex.

\subsection{Bison}
Bison

\subsection{Asm.js and Emscripten}
Asm.js is an emulated architecture written in Javascript and designed for execution within a web browser. Emscripten, a Mozilla project, is a backend for LLVM, allowing any LLVM-powered compiler to target the Asm.js architecture, resulting in code that can run in a web browser.

\subsection{Chrome Portable Native Client (PNaCL)}

\subsection{Web and Application Frameworks}
Frameworks like Play or Ruby on Rails 

%this is how to do an unnumbered subsection
\subsection*{Acknowledgements}
Other people did stuff too.

\begin{thebibliography}{9}

\bibitem{key:scala}
Martin Odersky, 
``The Scala Language Specification''

%\bibitem{foo:baz}
%A. N. Expert, 
%{\em A Book He Wrote,}
%His Publisher, 1989.

\end{thebibliography}

\end{document}